%% 
%% Copyright 2007, 2008, 2009 Elsevier Ltd
%% 
%% This file is part of the 'Elsarticle Bundle'.
%% ---------------------------------------------
%% 
%% It may be distributed under the conditions of the LaTeX Project Public
%% License, either version 1.2 of this license or (at your option) any
%% later version.  The latest version of this license is in
%%    http://www.latex-project.org/lppl.txt
%% and version 1.2 or later is part of all distributions of LaTeX
%% version 1999/12/01 or later.
%% 
%% The list of all files belonging to the 'Elsarticle Bundle' is
%% given in the file `manifest.txt'.
%% 

%% Template article for Elsevier's document class `elsarticle'
%% with numbered style bibliographic references
%% SP 2008/03/01

\documentclass[preprint,12pt, a4paper]{elsarticle}

%% Use the option review to obtain double line spacing
%% \documentclass[authoryear,preprint,review,12pt]{elsarticle}

%% For including figures, graphicx.sty has been loaded in
%% elsarticle.cls. If you prefer to use the old commands
%% please give \usepackage{epsfig}

%% The amssymb package provides various useful mathematical symbols
\usepackage{amssymb}
\usepackage{color}
\usepackage{hyperref}
\setlength{\parindent}{0pt}
%% The amsthm package provides extended theorem environments
%% \usepackage{amsthm}

%% The lineno packages adds line numbers. Start line numbering with
%% \begin{linenumbers}, end it with \end{linenumbers}. Or switch it on
%% for the whole article with \linenumbers.
%\usepackage{lineno}

\journal{SoftwareX}

\begin{document}
\renewcommand{\labelenumii}{\arabic{enumi}.\arabic{enumii}}

\begin{frontmatter}

%% Title, authors and addresses

%% use the tnoteref command within \title for footnotes;
%% use the tnotetext command for the associated footnote;
%% use the fnref command within \author or \address for footnotes;
%% use the fntext command for the associated footnote;
%% use the corref command within \author for corresponding author footnotes;
%% use the cortext command for the associated footnote;
%% use the ead command for the email address,
%% and the form \ead[url] for the home page:
%% \title{Title\tnoteref{label1}}
%% \tnotetext[label1]{}
%% \author{Name\corref{cor1}\fnref{label2}}
%% \ead{email address}
%% \ead[url]{home page}
%% \fntext[label2]{}
%% \cortext[cor1]{}
%% \address{Address\fnref{label3}}
%% \fntext[label3]{}

\title{gamma\_flow: \textbf{G}uided \textbf{A}nalysis of \textbf{M}ulti-label spectra by \textbf{Ma}trix \textbf{F}actorization for \textbf{L}ightweight \textbf{O}perational \textbf{W}orkflows}

%% use optional labels to link authors explicitly to addresses:
%% \author[label1,label2]{}
%% \address[label1]{}
%% \address[label2]{}

\author[ki-lab]{Viola Rädle \corref{cor1}}
\author[ki-lab]{Tilman Hartwig}
\author[ki-lab]{Benjamin Oesen}
\author[bfs]{Emily Alice Kröger}
\author[bfs]{Julius Vogt}
\author[bfs]{Eike Gericke}
\author[bfs]{Martin Baron}

\address[ki-lab]{Application Lab for AI and Big Data, German Environmental Agency, Leipzig, Germany}
\address[bfs]{Federal Office for Radiation Protection, Berlin, Germany}

\cortext[cor1]{Corresponding author: raedle.htwk@web.de}



\begin{abstract}
\catcode`\_=12
\textbf{gamma_flow} is an open-source Python package for real-time analysis of spectral data. It supports classification, denoising, decomposition, and outlier detection of both single- and multi-component spectra. Instead of relying on large, computationally intensive models, it employs a novel supervised approach to non-negative matrix factorization (NMF) for dimensionality reduction. This ensures a fast, efficient, and adaptable analysis while reducing computational costs. \textbf{gamma_flow} achieves classification accuracies above 90\% and enables reliable automated spectral interpretation. Originally developed for gamma-ray spectra, it is applicable to any type of one-dimensional spectral data. As an open and flexible alternative to proprietary software, it supports various applications in research and industry. 
\end{abstract}

\begin{keyword}
Python \sep Gamma spectroscopy \sep Non-negative Matrix Factorization \sep Classification \sep Denoising \sep Spectral Deconvolution

%% PACS codes here, in the form: \PACS code \sep code

%% MSC codes here, in the form: \MSC code \sep code
%% or \MSC[2008] code \sep code (2000 is the default)

\end{keyword}

\end{frontmatter}

%\linenumbers

\section*{Metadata}
\label{}
\textit{The ancillary data table~\ref{codeMetadata} is required for the sub-version of the codebase. Please replace the italicized text in the right column with the correct information about your current code and leave the left column untouched.}

\begin{table}[!h]
\begin{tabular}{|l|p{6.5cm}|p{6.5cm}|}
\hline
\textbf{Nr.} & \textbf{Code metadata description} & \textbf{Metadata} \\
\hline
TO DO C1 & Current code version & For example v42 \\
\hline
C2 & Permanent link to code/repository used for this code version & \url{https://gitlab.opencode.de/uba-ki-lab/gamma_flow} \\
\hline
C3  & Permanent link to Reproducible Capsule & TO DO For example: \url{https://codeocean.com/capsule/0270963/tree/v1}\\
\hline
C4 & Legal Code License   & BSD 3-Clause "New" or "Revised" License \\
\hline
C5 & Code versioning system used & git\\
\hline
C6 & Software code languages, tools, and services used & Python \\
\hline
C7 & Compilation requirements, operating environments \& dependencies & TO DO. Jupyter?  \\
\hline
C8 & If available Link to developer documentation/manual & \url{https://gitlab.opencode.de/uba-ki-lab/gamma_flow/-/blob/main/README.md?ref_type=heads} \\
\hline
C9 & Support email for questions & raedle.htwk@web.de\\
\hline
\end{tabular}
\caption{Code metadata (mandatory)}
\label{codeMetadata} 
\end{table}

\textcolor{red}{\textit{Optionally, you can provide information about the current executable
software version filling in the left column of
Table~\ref{executabelMetadata}. Please leave the first column as it is.} FRAGE: Welche Tabelle sollen wir ausfüllen?}

\begin{table}[!h]
\begin{tabular}{|l|p{6.5cm}|p{6.5cm}|}
\hline
\textbf{Nr.} & \textbf{(Executable) software metadata description} & \textbf{Please fill in this column} \\
\hline
S1 & Current software version & For example 1.1, 2.4 etc. \\
\hline
S2 & Permanent link to executables of this version  & For example: \url{https://github.com/combogenomics/DuctApe/releases/tag/DuctApe-0.16.4} \\
\hline
S3  & Permanent link to Reproducible Capsule & \\
\hline
S4 & Legal Software License & List one of the approved licenses \\
\hline
S5 & Computing platforms/Operating Systems & For example Android, BSD, iOS, Linux, OS X, Microsoft Windows, Unix-like , IBM z/OS, distributed/web based etc. \\
\hline
S6 & Installation requirements \& dependencies & \\
\hline
S7 & If available, link to user manual - if formally published include a reference to the publication in the reference list & For example: \url{http://mozart.github.io/documentation/} \\
\hline
S8 & Support email for questions & \\
\hline
\end{tabular}
\caption{Software metadata (optional)}
\label{executabelMetadata} 
\end{table}


\section{Motivation and significance}
\textit{In this section, we want you to introduce the scientific background and the motivation for developing the software.}

\begin{itemize}
    \item \textit{Explain why the software is important and describe the exact (scientific) problem(s) it solves.}
    \item \textit{Indicate in what way the software has contributed (or will contribute in the future) to the process of scientific discovery; if available, please cite a research paper using the software.}
    \item \textit{Provide a description of the experimental setting. (How does the user use the software?)}
    \item \textit{Introduce related work in literature (cite or list algorithms used, other software etc.).}
\end{itemize}

\section{Software description}

\textit{Describe the software. Provide enough detail to help the reader understand its impact. }

\subsection{Software architecture}
\textit{  Give a short overview of the overall software architecture; provide a pictorial overview where possible; for example, an image showing the components. If necessary, provide implementation details.}

 \subsection{Software functionalities}
\textit{  Present the major functionalities of the software.}
  
 \subsection{Sample code snippets analysis (optional)}


\section{Illustrative examples}

\textit{Provide at least one illustrative example to demonstrate the major
functions of your software/code.}

\textit{\textbf{Optional}: you may include one explanatory  video or screencast that will appear next to your article, in the right hand side panel. Please upload any video as a single supplementary file with your article. Only one MP4 formatted, with 150MB maximum size, video is possible per article. Recommended video dimensions are 640 x 480 at a maximum of 30 frames / second. Prior to submission please test and validate your .mp4 file at  \url{http://elsevier-apps.sciverse.com/GadgetVideoPodcastPlayerWeb/verification} . This tool will display your video exactly in the same way as it will appear on ScienceDirect. }

\textcolor{blue}{
Plots: 
\begin{itemize}
	\item Loadings
	\item Scores (different detector, single-label class.)
	\item Confusion matrix (different detector, single-label class.)
	\item Denoised spectrum
	\item Outlier: Feature importance (decision tree)
\end{itemize}
Problem: Diese Figures wollten wir eigentlich erst im wiss. Paper bringen! 
Alternative: Nur graphical abstract? Wirkt auf mich zu wenig für diese Section...
}

\section{Impact}
\textit{This is the main section of the article and reviewers will weight it appropriately.
Please indicate:}
\begin{itemize}
    \item \textit{Any new research questions that can be pursued as a result of your software.}
    \item \textit{In what way, and to what extent, your software improves the pursuit of existing research questions.}
    \item \textit{Any ways in which your software has changed the daily practice of its users.}
    \item \textit{How widespread the use of the software is within and outside the intended user group (downloads, number of users if your software is a service, citable publications, etc.).}
    \item \textit{How the software is being used in commercial settings and/or how it has led to the creation of spin-off companies.}
    \end{itemize}
\textit{Please note that points 1 and 2 are best demonstrated by
  references to citable publications.}

\section{Conclusions}
\cite{Kulesza2022}

\section*{Acknowledgements}
\label{}
\textit{Optional. You can use this section to acknowledge colleagues who don’t qualify as a co-author but helped you in some way. }


%% The Appendices part is started with the command \appendix;
%% appendix sections are then done as normal sections
%% \appendix

%% \section{}
%% \label{}

%% References:
%% If you have bibdatabase file and want bibtex to generate the
%% bibitems, please use
%%
%%  \bibliographystyle{elsarticle-num} 
%%  \bibliography{<your bibdatabase>}

%% else use the following coding to input the bibitems directly in the
%% TeX file.
\bibliographystyle{elsarticle-num} 
\bibliography{gamma_flow_SoftwareX.bib}
\begin{thebibliography}{00}

%% \bibitem{label}
%% Text of bibliographic item

\bibitem{} Use this style of ordering. References in-text should also use a similar style.

\end{thebibliography}

\textit{If the software repository you used supplied a DOI or another
Persistent IDentifier (PID), please add a reference for your software
here. For more guidance on software citation, please see our guide for
authors or \href{https://f1000research.com/articles/9-1257/v2}{this
  article on the essentials of software citation by FORCE 11}, of
which Elsevier is a member.}

\large{\textbf{Reminder: Before you submit, please delete all 
the instructions in this document, 
including this paragraph. 
Thank you!}}




\end{document}
\endinput
%%
%% End of file `SoftwareX_article_template.tex'.

%%% Local Variables:
%%% mode: latex
%%% TeX-master: t
%%% End:
